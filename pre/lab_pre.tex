\documentclass[UTF8]{beamer}
\usepackage{ctex}
%\usepackage{listings}
%\usepackage{fontspec}
%\usepackage{ulem}
%\usepackage{xcolor}
\usepackage[T1]{fontenc}
\usepackage{helvet}
\usepackage{tikz}
\usepackage{lipsum}
\usepackage{pgf}
\usepackage[english]{babel}
%\vspace{20cm}
\usetheme{Hannover}
%\usecolortheme{rose}


\title{MATH285 Lab project}

\author{Team 10}

\institute{\large ZJU-UIUC Institute}

\date{Spring Semester 2024}
\selectlanguage{english}
\begin{document}
% Title page
\begin{frame}
\titlepage
\begin{tikzpicture}[overlay,remember picture]
	\node(1)[xshift=3.3cm,yshift=-0.3cm] at (current page.center) {\includegraphics[scale=0.1]{Illi.png}};
	\node(2)[xshift=-1.8cm,yshift=-0.3cm] at (current page.center) {\includegraphics[scale=0.1]{zju.png}};
\end{tikzpicture}
\end{frame}

% \section{Outline}
% \begin{frame}
% \frametitle{Outline}
% \tableofcontents 
% \end{frame}

\begin{frame}[allowframebreaks]
\frametitle{Euler's Method}
%insert euler1.pgf
The Euler's method is based on the simple idea of approximating the solution of the ODE by the tangent line to the solution at the current point.
\begin{block}{Algorithm}
define IVP $f(t,y)$: $t = t_0$ and $y = y_0$ \\
input step size $h$ \\number of steps $n$ \\
for j from $1$ to $n$ do\\
$f_n=f(t,y)$\\
$y = y + h \cdot f_n$\\ $
t=t+h$
\end{block}
	\begin{figure}
		\centering
		\scalebox{0.45}{\input{euler.pgf}}
		\caption{Euler's Method for IVP1(left), IVP2(right)}
	\end{figure}
\end{frame}	

\begin{frame}[allowframebreaks]
	\frametitle{Improved Euler's Method}
The Improved Euler's method is based on the idea of using the average of the slopes at the beginning and the end of the interval to approximate the solution.
\begin{block}{Algorithm}
define IVP $f(t,y)$: $t = t_0$ and $y = y_0$ \\
input step size $h$ \\number of steps $n$ \\
for j from $1$ to $n$ do\\
$y = y + h \cdot \frac{f(t_n,y_n)+f(t_{n+1},y_{n+1})}{2}$\\ $
t=t+h$
\end{block}
	%insert euler1.pgf
	\begin{figure}
		\centering
		\scalebox{0.45}{\input{impeuler.pgf}}
		\caption{Improved Euler's Method for IVP1(left), IVP2(right)}
	\end{figure}
\end{frame}	

\begin{frame}[allowframebreaks]
	\frametitle{Runge-Kutta Method}
	%insert euler1.pgf
The Runge-Kutta method(a.k.a fourth-order four-stage Runge-Kutta method) is based on the idea of using the weighted average of the slopes at different points in the interval to approximate the solution.
\begin{block}{Algorithm}
define IVP $f(t,y)$: $t = t_0$ and $y = y_0$ \\
input step size $h$ \\number of steps $n$ \\
for j from $1$ to $n$ do\\
$y = y + h \cdot \frac{k_1+2k_2+2k_3+k_4}{6}$\\ $
t=t+h$\\
where $k_1=f(t_n,y_n), k_2=f(t_n+h/2,y_n+hk_1/2), k_3=f(t_n+h/2,y_n+hk_2/2), k_4=f(t_n+h/2,y_n+hk_3/2)$
\end{block}
	\begin{figure}
		\centering
		\scalebox{0.45}{\input{runge.pgf}}
		\caption{Runge-Kutta Method for IVP1(left), IVP2(right)}
	\end{figure}
\end{frame}

\begin{frame}[allowframebreaks]
	\frametitle{Adams-Bashforth Method}
	%insert euler1.pgf
The Adams-Bashforth method is based on the idea of using multistep method to approximate the solution.
	\begin{block}{Algorithm}
		define IVP $f(t,y)$: $t = t_0$ and $y = y_0$ \\
		input step size $h$ \\number of steps $n$ \\
		for j from $1$ to $n$ do\\
		$y = y + h \cdot \frac{55f_n-59f_{n-1}+37f_{n-2}-9f_{n-3}}{24}$\\ $
		t=t+h$
		\end{block}
	\begin{figure}
		\centering
		\scalebox{0.45}{\input{abm.pgf}}
		\caption{Adams-Bashforth Method for IVP1(left), IVP2(right)}
	\end{figure}
\end{frame}
\end{document}